\subsection{Opis zadania}

\quad W~zadaniu należy obliczyć wartość pochodnej funkcji, używając najpierw wzoru na różnicę prawostronną:

\begin{equation}
	f'(x) \approx \frac{f(x+h)-f(x)}{h},
	\label{zad1:eqn1}
\end{equation}
a~następnie używając wzoru różnic centralnych, czyli

\begin{equation}
	f'(x) \approx \frac{f(x+h)-f(x-h)}{2h}.
	\label{zad1:eqn2}
\end{equation}
Rozważaną funkcją jest $tan(x)$ w~punkcie $x=1$, natomiast przyjmowane wartości $h$~wynoszą $10^{-k}, k=0,1,...,16$.

Kolejnym etapem jest wyznaczenie błędu metody, błędu numerycznego i~błędu obliczenowego w~zależności od $h$~oraz odpowiednie ich przedstawienie na wspólnym wykresie i~wyciągnięcie wniosków.

Do obliczenia wspomnianych błędów pomocne będzie skorzystanie z~tożsamości 

\begin{equation}
	tan'(x)=1+tan^2(x),
	\label{zad1:eqn3} 
\end{equation}
która pozwala wyznaczyć rzeczywistą wartość pochodnej.
