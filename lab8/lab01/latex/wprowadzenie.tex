\quad Podczas tego laboratorium zrealizujemy 2 zadania, na podstawie których będziemy przeprowadzać analizę błędów - jest ich bowiem wiele rodzajów. 

W zadaniu pierwszym skupimy się na błędach obliczeniowych, numerycznych i~błędach metody. Przedstawimy je na przykładzie obliczania pochodnej wybranej funkcji w~punkcie, przybliżając ją za pomocą ilorazu różnicowego i~wykonamy obliczenia dla różnych wartości kroku. Skorzystamy z~odpowiednich zależności, aby obliczyć błędy i~zaprezentujemy je na wspólnym wykresie. 

Z kolei zadanie drugie polega na obliczania kolejnych wyrazów ciągu rekurencyjnego zdefiniowanego równaniem różnicowym, przyjmując różne reprezentacje liczb zmiennoprzecinkowych. Dla każdej z~nich przeanalizujemy otrzymane wartości wyrazów ciągu i~porównamy z~teoretycznymi, następnie przedstawimy odpowiednio błędy względne w~każdej z~reprezentacji.