\subsection{Opis zadania}

\quad Celem tego ćwiczenia jest zrozumienie sposobu rozwiązywania równań różnicowych przy użyciu iteracji oraz analiza wpływu precyzji obliczeń na wyniki.
Rozważmy następujące równanie różnicowe:
\begin{align*}
        x_{k+1} = 2.25x_k - 0.5x_{k-1},\\
        \textrm{gdzie } x_0 = \frac{1}{3}, x_1 = \frac{1}{12}.
\end{align*}

Zadanie polega na wygenerowaniu pierwszych $n$ wyrazów tego ciągu oraz analizie zachowania się ciągu w zależności od precyzji użytych obliczeń.

Wykonamy obliczenia dla 3 przypadków:
\begin{enumerate}
    \item używając pojedynczej precyzji (float32) oraz przyjmując $n = 225$,
    \item używając podwójnej precyzji (float64) oraz przyjmując $n = 60$,
    \item używając reprezentacji \emph{Fraction} z~biblioteki \emph{fractions} oraz przyjmując $n = 225$.
\end{enumerate}

Dalej porównamy wyniki uzyskane przy użyciu wymienionych reprezentacji liczb zmiennoprzecinkowych z wartościami obliczonymi przy użyciu postaci jawnej ciągu:
$$
x_k = \frac{4^{-k}}{3}.
$$