\quad W ramach tego laboratorium przeprowadziliśmy analizę numerycznych metod obliczania pochodnej funkcji i~wyrazów ciągu rekurencyjnego oraz dokonaliśmy analizy błędów związanych z~tymi metodami.

W pierwszym zadaniu zaimplementowaliśmy numeryczną metodę obliczania pochodnej funkcji, używając przybliżenia za pomocą ilorazu różnicowego. Przetestowaliśmy tę metodę dla funkcji tangens oraz wyznaczyliśmy błąd, porównując wyniki numeryczne z~prawdziwą wartością pochodnej. Następnie przedstawiliśmy wykresy wartości bezwzględnej błędu metody, błędu numerycznego oraz błędu obliczeniowego w zależności od kroku $h$. Zauważyliśmy, że błąd obliczeniowy osiąga minimum dla określonej wartości $h$, co zgadza się z teoretycznymi przewidywaniami, jakie wynikają z analizy 
błędów numerycznych.

W drugim zadaniu zaimplementowaliśmy program generujący wyrazy ciągu zdefiniowanego równaniem różnicowym. Przeprowadziliśmy obliczenia dla różnych precyzji (pojedynczej, podwójnej oraz reprezentacji ułamków) i~narysowaliśmy wykresy wartości ciągu w~zależności od $k$. Dodatkowo, przedstawiliśmy wykres wartości bezwględnej błędu względnego w~zależności od $k$. 
Zaobserwowaliśmy, że wartości ciągu stale maleją wraz ze wzrostem $k$~jedynie dla reprezentacji fractions, a float32 i~float64 zachowują precyzję tylko dla ograniczonej ilości początkowych wyrazów.
